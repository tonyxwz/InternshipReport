Algorithm \ref{algo:rayboxinter} is implemented as a member function of the class \texttt{Box}.
This function is called for every ray in the model thus needs optimizing. The algorithm is only suitable for axis-aligned bounding boxes. 


\IncMargin{1em}
\begin{algorithm}[H] \label{algo:rayboxinter}
    \DontPrintSemicolon
    \SetKwInOut{Input}{Input}\SetKwInOut{Output}{Output}
    \Input{ray, box}
    \Output{two intersection points}
    tx1 = (box.min.x - ray.x0.x)/ray.n.x\;
    tx2 = (box.max.x - ray.x0.x)/ray.n.x\;
    tmin = min(tx1, tx2)\;
    tmax = max(tx1, tx2)\;

    ty1 = (box.min.x - ray.x0.x)/ray.n.x\;
    ty2 = (box.max.x - ray.x0.x)/ray.n.x\;
    tmin = max(tmin, min(ty1, ty2))\;
    tmax = min(tmax, max(ty1, ty2))\;

    tz1 = (box.min.x - ray.x0.x)/ray.n.x\;
    tz2 = (box.max.x - ray.x0.x)/ray.n.x\;
    tmin = max(tmin, min(tz1, tz2))\;
    tmax = min(tmax, max(tz1, tz2))\;

    \uIf{tmax > tmin}{
        \Return tmax, tmin
    }
    \Else{
        \Return None, None
    }
    \caption{Ray intersects with an axis-aligned bounding box.}
\end{algorithm}
\DecMargin{1em}