% !TeX root = ../report.tex

High Intensity Focused Ultrasound (HIFU) is a technique that uses non-ionizing ultrasonic waves to heat tissue the in human body. In the past two decades, it has been applied in treatment of a multitude of pathologic conditions. By applying high-energy ultrasound beams focused only at the region of interest, HIFU can deliver heat to the target without harming the surrounding tissue \cite{JENNE2012311}. Although the therapeutic application of ultrasound is less prevalent than the diagnostic application, it is rapidly gaining popularity and is at various stages of development and commercialization \cite{wp_HIFU}. The clinical and preclinical application of HIFU includes the thermal ablation of benign and maglinant lesions, targeted drug delivery through thermal-sensitive liposomes, the treatment of sonothrombolysis and so on. The simulation model in this study only focuses on the thermal ablation of bone metastases, which causes chronic and severe pain to patients and hence lower their life quality significantly. As a result, the media taken into consideration is only oil, muscle (modeled as fluid) and bone (solid). HIFU can remove the tissue that causes pain through high pressure on nerves. Clinical trials have already shown promising results, where the patient's pain is immediately relieved after treatment with HIFU\cite{vanwijk2013}.

When applying HIFU in therapy, doctors need visual guidance to help planning, controlling and monitoring the treatment process and to ensure the safety and efficacy of ultrasound exposure. Diagnostic Ultrasound-guided High Intensity Focused Ultrasound (USgHIFU) and Magnetic Resonance-guided High Intensity Focused Ultrasound (MR-HIFU) (Figure \ref{fig:HIFU_example}), are designed to provide the doctors with insightful information during the treatment process in nearly realtime. In this study, we focus only on MR-HIFU, since USgHIFU's signal to temperature mapping encounters a bifurcation for high temperature values MR-HIFU is the preferred method of temperature monitoring. In an MR-HIFU treatment, the MR-thermometry is to provide the temperature information for the doctors to plan the treatment. However, this technique is limited when bone tissue is involved. Magnetic Resonance performs better in the water-rich tissues which have a longer traversal relaxation time (T2) \cite{Modena_2018}. The concentration of water in bone is much lower than that of skin or muscle (22\% compared with 75\%). As a result, it has a shorter T2 and is harder to measure with high precision. What's making the problem even worse is that the temperature increase in bone tissue caused by ultrasound is faster than muscle tissue due to the different heat capacity.

\begin{figure}[h]
    \centering
    \includegraphics[width=\textwidth]{HIFU-example2}
    \caption{(A): an MRI-image of a tumorous uterus, (B): treatment planning for HIFU. The yellow is where the ultrasound is expected to pass, the orange is the ablation region, the purple is the target for ablation. (C) Real time temperature monitoring by MR-HIFU. The color indicates the increases in temperature. \cite{vanwijk2013}}
    \label{fig:HIFU_example}
\end{figure}

To solve this problem, computer simulation is introduced to help doctors estimate the temperature in bone tissue. Many methods have been developed to simulate HIFU systems \cite{StochasticSim}. Most of them use a ray tracing method, which treats the ultrasonic wave as ray beams propagating towards various media. However, these types of methods are usually time consuming as they require the ray tracer to cast a large number of rays so that the sampled intensity can converge to realistic values. In this report, a new ray tracer is proposed which keeps track of the divergence of the beam to estimate the intensity. The result shows that this new method can reduce the number of required rays and save some computational time.

\begin{figure}[h]
    \centering
    \includegraphics[width=0.5\textwidth]{box_transducer}
    \caption{Illustration of transducer elements casting ultrasound rays at a medium interface}
    \label{fig:transducer_coordinates}
\end{figure}

The study is divided into the following steps:

\begin{enumerate}
    \item Implement a new ray tracing method using trident rays
    \item Simulate a HIFU system with only one medium.
    \item Simulate a HIFU system with two fluid media.
    \item Parameter analysis of stem 3.
    \item Qualitative comparison of the new result to the old method's result.
\end{enumerate}

In the Methods section, a more detailed discussion about the theory and implementation will be presented. In the Results section, the output of the model in this study will be compared with the model of Modena et al \cite{Modena_2018}, based on similarity metrics.
